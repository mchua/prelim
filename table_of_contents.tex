* list
* list
** listy
** listy
* list

Intro
======

I propose an engineering education dissertation that contributes to the fledgling subfield of faculty development research.
----------------
* Missed opportunities for faculty development
* FacDev/SOTL is a thing; it's still sparse in Engr Edu

We can frame the situation as a language problem.
------------------------
* Engineering faculty don't speak "pedagogy" language
* We discuss curricular reform at either super-high theoretical levels or super-specific practitioner levels; there's not much in the middle linking theory to practice

We shouldn't be surprised by this language problem, because here's the current situation of engineering faculty.
---------------------------------
* New faculty are isloated and not prepared for teaching
* We socialize faculty into research, but not teaching
* There's not really a cohesive CoP for engineering teaching

What we know about learning does not match up with the situation.
------------------------
* Knowledge is situated and enculturated (epistemology)
* Apprentices progress from specific embedded activities to general cultural principles

Prior attempts to change this situation have had limited effect.
-------------------
* Validity: does it reach people? (Who? What format?)
* Research-to-practice does not happen with papers
* [Qualities of existing faculty development initiatives]

Again, we shouldn't be surprised by this lack of change, given what we know about change.
-------------------------------
* We resist change when we can't see ourselves in a changed future
* We change our stories about ourselves before we change our visible behavior
* Ontology - identity is performative

How does teaching change in engineering actually happen?
---------------
* Quality: does the research spark conversation? Does it get "used"?
* Teaching practice changes happen because of accidental bumps into people, not deliberate searches. This is how teaching change really happens.

Research question
------------------
* RQ: How do engineering and technology faculty make sense of their teaching identities through public storytelling centered on curricular revision?

My narrators and me: positionality and participants
====================================================

This project studies individual engineering and technology faculty members.
--------------
* Faculty are my unit of analysis
* [Description of participants] as a unit of study

My participants and I have an existing relationship dynamic.
---------------------
* participating faculty are "narrators" (not "participants" or "subjects")
* I'm a student and they're profs (power dynamic)
* I like my narrators
* Existing relationships with narrators

Why a poststructural approach
-------------------
* I take a poststructural approach of interruption and discomfort
* My themes are emergent because I want to deliberately coconstruct them and point out we are socially constructing them

The implications of a poststructural stance on the portrayal of positionality
------------------------
* I'm biased
* I'm opinionated on these topics of discusion
* I will try to shut up and work with/elicit narrator voices
* Get participants to claim agency and recognize it in others

Cognitive apprenticeship as a priori scaffolding
=================================================

Cognitive Apprenticeship is my theoretical starting point
---------------
* CogApp is my a priori theory
* CogApp is a temporary scaffolding; I'm using it until something better comes along, and expect to end up with something different
* Looking for opportunities to make CogApp fall apart (very poststructural of me)
* CogApp makes thinking visible in a community in action

Cognitive Apprenticeship codes
------------------------
* [list of CogApp codes]
* CogApp mentors model, scaffold, coach, & fade - metacognition support
* Mentors create ZPDs (for things like metacognition)

Reflection is a part of CogApp
-----------------------
* Reflective practice (which is related to metacognition) is an established thing with methods & benefits documented
* I'm asking my narrators to reflect on their experiences on curric rev teams
* We benefit from being conscious of things as we do them
* This reflective practice bridges past actions to present consciousness and future plans

Identity development and CogApp are related
------------------------
* Identity development is the outcome of CogApp - apprentices make themselves into masters
* Look, this addresses my research question
* I want to make CogApp happen between my narrators

Narrative analysis as methodology
==================================

The methodological choice of narrative analysis comes from discourse ideas within Cognitive Apprenticeship
--------------------
* I am using narrative analysis
* sensemaking is storytelling
* public storytelling on curricular revision is engaging in a cognitive apprenticeship
* discourse participation is an important CogApp skill

Individual narratives and (performed) identities are also communal in nature
-----------------------
* NarAn is a community engagement
* We write our stories together, we perform them to each other, we interpret them for each other, we disagree (poststructural)
* We naturally comment on each others' stories
* NarAn helps individual reflection in/on community roles
* Our autobiographical storytelling interacts with our identity formation
* We make sense of our (and others') teaching identities together

It's within this sort of community environment/engagement that individual identities develop
--------------------------
* Learners situate their own development within many living other examples of mastery-in-progress (older apprenticeships and journeymen on the open shop floor)
* Increased ability to make sense of context is a hallmark of mastery

Transparency
=============

I'm taking a radically transparent approach to my narrative analysis to provide the following affordances
------------------
* I do my NarAn in a structured and transparent way
* Transparency allows back-tracing (validity!)

This lets more people have access
-------------
* Lurking is key for underrepresented groups
* Legitimizing peripheral participation is key to draw in new participants

This dissertation is an experiment with a new research method
---------------
* RTR is a new method in qualitative research
* It's been piloted
* This dissertation is the next step and intended to provide a full example of its usage in a research project
* This may be a bigger contribution than the actual study outcomes, but we'll see
* I am using it right now, not studying it; don't get distracted
* Maybe I'll study it later

Step by step
=============

* [What narrators are asked to do]
* [data analysis walkthrough]
* Justifying dataset choices, saturation/validity, planning

So what?
=========

* [research outputs]
* [potential audiences for outputs]
* [project impact] hopefully solves problems listed at start

Appendices
===========

* Timeline
* Description of schools
* Interview setup logistics
* Interview "script"
* Datasets produced by this project
* Public publishing logistics
* List of stuff that's out of scope
